%\documentclass[sigconf, authordraft]{acmart}
\documentclass[sigconf]{acmart}

\usepackage{booktabs} % For formal tables


% Copyright
%\setcopyright{none}
%\setcopyright{acmcopyright}
%\setcopyright{acmlicensed}
\setcopyright{rightsretained}
%\setcopyright{usgov}
%\setcopyright{usgovmixed}
%\setcopyright{cagov}
%\setcopyright{cagovmixed}


% DOI
\acmDOI{10.1145/nnnnnnn.nnnnnnn}

% ISBN
\acmISBN{978-x-xxxx-xxxx-x/YY/MM}

% Conference
\acmConference[GECCO '19]{the Genetic and Evolutionary Computation Conference 2019}{July 13--17, 2019}{Prague, Czech Republic}
\acmYear{2019}
\copyrightyear{2019}

%\acmArticle{4}
\acmPrice{15.00}

% These commands are optional
%\acmBooktitle{Transactions of the ACM Woodstock conference}
%\editor{Jennifer B. Sartor}
%\editor{Theo D'Hondt}
%\editor{Wolfgang De Meuter}


\begin{document}
\title{Exploring concurrent and stateless evolutionary algorithms}
\titlenote{Produces the permission block, and
  copyright information}
\subtitle{Subtitle}
\subtitlenote{The full version of the author's guide is available as
  \texttt{acmart.pdf} document}

  \author{Anonymous Author 1}
  \orcid{1234-5678-9012}
  \affiliation{%
    \institution{Anonymous institute 1}
    \streetaddress{P.O. Box 1212}
    \city{Dublin}
    \state{Ohio}
    \postcode{43017-6221}
  }
  \email{aa1@ai1.com}
  
  \author{Anonymous Author 2}
  \affiliation{%
    \institution{Anonymous institute 2}
    \streetaddress{P.O. Box 1212}
    \city{Dublin}
    \state{Ohio}
    \postcode{43017-6221}
  }
  \email{2aa@ai2.com}

  \author{Anonymous Author 3}
  \affiliation{%
    \institution{Anonymous institute 2}
    \streetaddress{P.O. Box 1212}
    \city{Dublin}
    \state{Ohio}
    \postcode{43017-6221}
  }
  \email{2aa@ai2.com}

  \author{Anonymous Author 4}
  \affiliation{%
    \institution{Anonymous institute 2}
    \streetaddress{P.O. Box 1212}
    \city{Dublin}
    \state{Ohio}
    \postcode{43017-6221}
  }
  \email{2aa@ai2.com}
  
  % The default list of authors is too long for headers.
  \renewcommand{\shortauthors}{A. Author et al.}


\begin{abstract}
  Creating a concurrent and stateless version of an evolutionary algorithm implies
  changes in its behavior. From the performance point of view, the main challenge is to balance computation
  with communication, but from the algorithmic point of view we have to
  keep diversity high so that the algorithm is not stuck in local
  minima. In a concurrent setting, we will have to find the right
  balance so that improvements in both fields do not cancel out. This is
  what we will be doing in this paper, where we explore the space of
  parameters of a population based concurrent evolutionary algorithm to
  find out the best combination for a particular problem.\end{abstract}

%
% The code below should be generated by the tool at
% http://dl.acm.org/ccs.cfm
% Please copy and paste the code instead of the example below. 
%
\begin{CCSXML}
<ccs2012>
 <concept>
  <concept_id>10010520.10010553.10010562</concept_id>
  <concept_desc>Computer systems organization~Embedded systems</concept_desc>
  <concept_significance>500</concept_significance>
 </concept>
 <concept>
  <concept_id>10010520.10010575.10010755</concept_id>
  <concept_desc>Computer systems organization~Redundancy</concept_desc>
  <concept_significance>300</concept_significance>
 </concept>
 <concept>
  <concept_id>10010520.10010553.10010554</concept_id>
  <concept_desc>Computer systems organization~Robotics</concept_desc>
  <concept_significance>100</concept_significance>
 </concept>
 <concept>
  <concept_id>10003033.10003083.10003095</concept_id>
  <concept_desc>Networks~Network reliability</concept_desc>
  <concept_significance>100</concept_significance>
 </concept>
</ccs2012>  
\end{CCSXML}

\ccsdesc[500]{Computer systems organization~Embedded systems}
\ccsdesc[300]{Computer systems organization~Redundancy}
\ccsdesc{Computer systems organization~Robotics}
\ccsdesc[100]{Networks~Network reliability}


\keywords{ACM proceedings, \LaTeX, text tagging}


\maketitle

\section{Introduction}

Concurrent programming adds a layer of abstraction over the parallel
facilities processors and operating systems to offer a
high-level interface that allows the user to program code that might
be executed in parallel either in threads or in different processes \cite{andrews1991concurrent}.

Different languages offer different facilities for concurrency at the
primitive level, and mainly differ on how they deal with shared state,
that is, variables that are accessed from several
processes. Actor-based concurrency \cite{schippers2009towards}
eliminates shared state by introducing a series of {\em actors} that
store state and can change it; on the other hand,  channel based
concurrency follows the {\em communicating sequential processes}
methodology \cite{Hoare:1978:CSP:359576.359585}, which is effectively
stateless, with different processes reacting to channel input without
changing state, and writing to these channels. 

This kind of concurrency is the one implemented by many modern
languages such as Go or Perl 6 \cite{lenzperl}. However, the
statelessness of the implementation asks for a change in the
implementation of any algorithm. In particular, evolutionary
algorithms need to change to fit an architecture that creates and
processes streams of data using functions that do not change state. 

Despite the emphasis on hardware-based techniques such as
cloud computing or GPGPU, there are not many papers \cite{Xia2010} dealing with
creating concurrent evolutionary algorithms that work in a single
computing node or that extend seamlessly from single to many computers.

For instance, the EvAg model \cite{evag:gpem} is a locally concurrent and globally
parallel evolutionary algorithm that leaves the management of the
different agents (i.e. threads) to the underlying platform scheduler
and displays an interesting feature: the model is able to scale
seamlessly and take full advantage of CPU threads. In a first attempt
to measure the scalability of the approach experiments were conducted
in \cite{wcci:evoag} for a single and a dual-core processor showing
that, for cost functions passing some milliseconds of computing
time, the model was able to achieve near linear speed-ups . This study
was later on extended in \cite{DBLP:conf/evoW/LaredoBMG12} by scaling
up the experimentation to up to 188 parallel machines. The
reported speed-up was $\times 960$ which is beyond the linear $\times
188$ that could be expected if local concurrency were not taken into
account.  

The aforementioned algorithm used a protocol that worked
asynchronously, leveraging its peer-to-peer capabilities; in general
the design of concurrent EAs has to take into account the
communication/synchronization 
between processes, which nowadays will be mainly threads. Although the
paper above was original in its approach, other authors targeted
explicitly multi-core architectures, such as Tagawa
\cite{Tagawa201212} which used shared memory and a clever mechanism to
avoid deadlock. Other authors \cite{kerdprasop2012concurrent} actually
use a message-based architecture based in the concurrent functional
language Erlang, which separates GA populations as different
processes, although all communication takes place with a common
central thread. 

In our previous papers 
\cite{Merelo:2018:MEA:3205651.3208317:anon,Garcia-Valdez:2018:MEA:3205651.3205719:anon},
we presented the proof of concept and initial results with this kind
of stateless evolutionary algorithms, implemented in the Perl 6
language. These evolutionary algorithms use a single channel where
whole populations are sent. The (stateless) functions read a single
population from the channel, run an evolutionary algorithm for a fixed
number of generations, which we call the {\em generation gap} or
simply {\em gap}, and send the population in the final generation back
to the channel. Several populations are created initially, and a
concurrent {\em mixer} is run which takes populations in couples,
mixes them leaving only a single population with the best individuals
selected from the two merged populations.
 This {\em
  gap} is then conceptually, if not functionally, similar to the {\em
  time to migration} in parallel evolutionary algorithms (with which
concurrent evolutionary algorithms have a big resemblance).

We did some initial exploration of the parameter space in
\cite{merelo:WEA:anon}. In these initial explorations we realized
that the parameters we used had an influence at the algorithmic level,
but also at the implementation level, changing the wallclock
performance of the algorithm.

In this paper we will explore the parameter space systematically
looking particularly at two parameters that have a very important
influence on performance: population size and generation gap. Our
intention is to create a rule of thumb for setting them in this kind
of algorithms, so that they are able to achieve the best
performance. 

We will present the experimental setup next.

\section{Experimental setup and results}
\label{sec:res}
In principle, we will work with two threads, one for
mixing the populations and another for evolution. This setup, while
being concurrent, allows us to focus more on the basic features of the
algorithm, which is what we are exploring in this paper. 

What we want to find out in these set of experiments is what is the
generation gap that gives the best performance in terms of raw time to
find a solution, as well as the best number of evaluations per
second. In order to do that, we prepared an experiment using the
OneMax function with 64 bits, a concurrent evolutionary algorithm such
as the one described in \cite{Merelo:2018:MEA:3205651.3208317:anon},
which is based in the free Perl 6 evolutionary algorithm library {\tt
  Algorithm::Evolutionary::Simple}, and run the experiments in a
machine with Ubuntu 18.04, an AMD Ryzen 7 2700X Eight-Core Processor
at 2195MHz. The Rakudo version was 6.d, which had recently been
released with many improvements to the concurrent core of the
language. All scripts, as well as processing scripts and data obtained
in the experiments are available, under a free license, from our GitHub
repository.

%
%  Figure with Plots goes here
%

We used a population size of 256, as well as generation gaps
increasing from 8 to 64. Many experiments were run for every
configuration, up to 150 in some cases. We logged the upper bound of
the number of evaluations needed (by multiplying the number of
messages by the number of generations and number of individuals
evaluated; this means that this number will be an upper bound, and not
the exact number of evaluations until a solution is reached). We will
first look at the general picture by plotting the wallclock time in seconds
(measured by taking the time of the starting of the algorithm and the
last message and subtracting the latter from the former) vs the
number of evaluations that have been performed. The result is shown in
Figure \ref{fig:evals}. Experiments with different generation gaps are
shown with different colors (where available) and shapes, and they
spread in an angle which is roughly bracketed by the experiments with
a generation gap of 8, which need the most time for the same number of
evaluations, and the experiments with a gap of 16, which usually need
the least. The experiments with gaps = 32 or 64 are somewhere in
between.

In that same chart it can also be observed that the number of
evaluations needed to find the 64 bit OneMax solution is quite
different. We make a boxplot of the number of evaluations vs the
generation gap in Figure \ref{fig:evals:bp}. This figure shows an
increasing number of evaluations per gap size. Differences are
significant between every generation gap and the next. This
increasing number of evaluations per generation gap is probably due to
the fact that the increasing number of isolated generations makes the
population lose diversity, making finding the solution increasingly
difficult. This is the same effect observed in parallel algorithms, as
reported in \cite{Cantu-Paz:1999:MPT:2933923.2934003}, so it is not
unexpected. What is unexpected is the combination of generation gap
size and the concurrent algorithm, since it is impossible to know in
advance what is the optimal computation to communication balance.
% Here it would be interesting to describe why is unexpected? - Juanlu
% done - JJ

We plot the number of evaluations per second in Figure
\ref{fig:evals:avg}. These show a big difference for a generation gap
of 16, with a number of evaluations which is almost 50\% higher than
for the rest of the generation gaps, where the difference is not so
high. 


The number of evaluation per second does not follow a clear trend. It
falls and remains flat for a generation gap higher than 16; it is also slightly higher than for the minimum
generation gap that has been evaluated, 8. This generation gap,
however, presents also the lowest number of evaluations to solution,
which means that, on average, the solution will be found faster with a
generation gap of 8 or 16. This is shown in Figure \ref{fig:time}.
% I don't feel confortable modifying this paragraph, but I find it a bit twisted - Juanlu
% Rewritten - JJ

\section{Conclusions}
\label{sec:conclusions}
In this paper we have set out to explore the interaction between the
generation gap and the algorithmic parameters in a concurrent and stateless evolutionary algorithm. From the point of view
of the algorithm, increasing the generation gap favors exploitation
over exploration, which might be a plus in some problems, but also
decreases diversity, which might lead to premature convergence; in a
parallel setting, this will make the algorithm need more evaluations
to find a solution. The effect in a concurrent program goes in the
opposite direction: by decreasing communication, the amount of
code that can be executed concurrently increases, increasing
performance. Since the two effects cancel out, in this paper we have
used a experimental methodology to find out what is the combination
that is able to minimize wallclock time, which is eventually what we
are interested in by maximizing the number of evaluations per second
while, at the same time, increasing by a small quantity the number of
evaluations needed to find the solution.

For the specific problem we have used in this short paper, a 64-bit
onemax, the generation gap that is in that area is 16. The time to
communication for that specific generation gap is around 2 seconds,
since 16 generations imply 4096 evaluations and evaluation speed is
approximately 2K/s. This gives us a ballpark of the amount of
computation that is needed for concurrency to be efficient. In this
case, we are sending the whole population to the communication
channel, and this implies a certain overhead in reading, transmiting
and writing. Increasing the population size also increases that
overhead.

We can thus deduce than the amount of computation, for this particular
machine, should be on the order of 2 seconds, so that it effectively
overcomes the amount of communication needed. This amount could be
played out in different way, for instance by increasing the
population; if the evaluation function takes more time, different
combinations should be tested so that no message is sent unless that
particular amount of time is reached.

With these conclusions in mind, we can set out to work with other
parameters, such as population size or number of initial populations,
so that the loss of diversity for bigger population sizes is
overcome. Also we have to somehow overcome the problem of the message
size by using a statistical distribution of the population, or simply
other different setup. This is left as future work.%\end{document}  % This is where a 'short' article might terminate



\begin{acks}
Acknowledgements taking\\
this much\\
space

\end{acks}

\bibliographystyle{ACM-Reference-Format}
\bibliography{../geneura,../concurrent,../perl6} 

\end{document}
